
\chapter{Introduction to Project}\hrule
\label{Chapter:1}
% =====================================================================================================
\section{Overview}

Scisso App aims to solve problems of the Salon Managment and marketing across the country. The main motive of Scisso App is keeping the retention rate of the customers as high as possible by providing offers and loyalty points to the user for each purchase they make. This app will help Salon Admin to keep track of their product delivery, customer retention and Employee efficiency.\\
\\
While developing the project, i followed the industry standards of MVC i.e Model View Controller model. It follows the software engineering principles of low coupling and high cohesion. This model is followed to develop better softwares and to maintain a sustianable software product for a longer period of time.\\
\\
\textbf{What is MVC model?}\\
The MVC pattern allows separate the Controller layer from the logic, so that everything about how the interface works is separated from how we represent it on screen. \\
\\
Ideally the MVC pattern would achieve that same logic might have completely different and interchangeable views.\\
\\
\textbf{Why use MVC?}\\
In Android we have a problem arising from the fact that Android activities are closely coupled to both interface and data access mechanisms. We can find extreme examples such as Adapters, which are part of the view, with cursors, something that should be relegated to the depths of data access layer .

For an application to be easily extensible and maintainable we need to define well separated layers. What do we do tomorrow if, instead of retrieving the same data from a database, we need to do it from a web service? We would have to redo our entire view .

MVC makes views independent from our data source. We divide the application into at least three different layers, which let us test them independently. With MVC we are able to take most of logic out from the activities so that we can test it without using instrumentation tests.\\

While developing the project the main focus was that how the app will keep track of its customers and their transactions. Different API and their standard methods has been used to ensure that the app works superbly. Different latest android concepts like CardView also have been used in the project in order to have it a beautiful UI.\\
\\
The app is a total solution for the managment and marketing of a Slaon. The app allows the admin to handle the services provoided by them. The services are arranged according to the their categories. Along with services admin is allowed to handle and edit the data of their clients. Along with these facilities app also saves and have track of the employees working in the salon. Along with all these app also counts the incentives gained by the employees. These modules were from the sofware solutions that are provided by the app. The app also provide the Salon with marketing solution by rolling offers and by giving loyalty points on every purchase a customer make. 
\\

Future of the app is to bring all the data of every salon using the app at a single database which will contain the data of all the clients from all the salon's and after gathering all the data we can analyse them and run acccording ads to the particular customer.

\section{Existing System}
No such type of Independent App exists for the Salon app managment which can provide both of the Software and marketing solutions. Though there are some of the apps like Vaniday which are providing great software soltions to the Salon's.

\chapter{Feasibility Study}\hrule
A feasibility study is used to determine the viability of an idea, such as ensuring a project is legally and technically feasible as well as economically justifiable. It tells us whether a project is worth the investment—in some cases, a project may not be do-able. There can be many reasons for this, including requiring too many resources, which not only prevents those resources from performing other tasks but also may cost more than an organization would earn back by taking on a project that isn’t profitable.\\
\\
The application is fully feasible. It just needs a working internet connection and android 4.0 and above. It is fully feasible if it is also deployed on a large scale.\\
The application can also be upgraded further and can be deployed on large scale depending upon the need of business plan.\\
\begin{enumerate}
	\item \textbf{Technical Feasibility :} The app is fully feasible on technical terms. I have android studio and required 8 GB Ram for development purpose.\\
	I will use Firebase cloud to deploy backend as it is free for inital and small projects.\\
	The version control system is completely free and the website Github.com is also free for Open Source Projects.
	\item \textbf{Economic Feasibility :} The app is fully economically feasible as it has free and open source tools being used while developing the system.\\
	The Firebase Cloud technology is free till 10K hits in a day. So, intially around 6-7 months after releasing the project, it is expected to have a user base of around 10K people. Later on, If the user base increases, the revenue of product will also increase and further resources can be purchased.
	\item \textbf{Legal Feasibility :} The app doesn't violates any legal rights and will credit the author of open Source Library used while developing the project.\\
	The project will be available in open source under \textbf{GPLv3} license.\\
	\\
	\textbf{What is GPLv3 License?}
	\begin{itemize}
		\item The source code must be made public whenever a distribution of the software is made.
		\item Modifications of the software must be released under the same license.
		\item Changes made to the source code must be documented.
		\item If patented material was used in the creation of the software, it grants the right for users to use it. If the user sues anyone over the use of the patented material, they lose the right to use the software.\\
	\end{itemize}
	
	\item \textbf{Operational Feasibility : } As the app satisfies the functional and non functional requirements, the app will be fully operational once it releases.
	
	\item \textbf{Scheduling Feasibility : } The project release targets for different versions are practical and have plenty of time develop and debug the app before release.
\end{enumerate}

\chapter {Objectives of the Project }\hrule
\begin{itemize}
	\item The project is developed to give a software solution and a marketing solution to the Salon Admin.
	\item The main motive behind developing the project is to rescue the Salon from hectic calculations. The app will help Salon to increase their revenue.
	\item This app will provide us with a large database of the customers which can be used to analyse and run ads accordingly.
\end{itemize}


\chapter{Assumptions and Dependencies}\hrule
\begin{itemize}
	\item Admin is assumed to be using Android 4.4 and above.
	\item Admin is assumed to have working internet connection.
	\item Incentives and loyalty points are dependent on the respective rate.
	\item Date is dependent on system time. 
	
\end{itemize}

\chapter{Specific Requirements}\hrule
\begin{itemize}
	\item Android 4.4 and above.
	\item Working internet connection.
	\item Minimum 1 GB Ram and 2 GB external memory.
	